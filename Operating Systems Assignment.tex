\documentclass{article}

\usepackage{listings}
\usepackage{xcolor}

% https://tex.stackexchange.com/questions/116534/lstlisting-line-wrapping
\lstset{
    language=C,
    frame=single,
    breaklines=true,
    postbreak=\mbox{\textcolor{red}{$\hookrightarrow$}\space},
}

\title{KV5002 Assignment}
\author{w20013000}
\date{}

\begin{document}

\maketitle

\section{OS Theory}
    \subsection{What is a multiprocessing system?}
    // TODO: This
    \subsubsection{Examples}
    // TODO: This
    \subsubsection{Benefits}
    // TODO: This

\subsection{The difference between a process and a thread?}
A process is an instance of a program currently running on the system.
A thread, meanwhile, is the smallest unit of execution scheduled by the operating system.
Each thread belongs to a process, and each process has one or more threads.

Different threads within the same process share the same memory which includes any
statically or dynamically allocated variables. Each thread has its own stack and registers
which store local variables.
Different processes have separate memory spaces, even if they are running the same program.

A process that requires multiprocessing will typically create multiple threads
rather than multiple processes as creating and communicating between processes is
more expensive than with threads.

    \subsubsection{Scenario 1}
    You're writing a report in Microsoft Office. While you are typing, the text is
    displayed, and spelling and grammar are checked.

    This is an example of a process using multiple threads.
    One thread is responsible for displaying the text, while another thread is responsible
    for checking spelling and grammar in it.

    Both threads need to access the same resource, the text being written, and therefore
    need to be part of the same process to share the same memory space.

    \subsubsection{Scenario 2:}
    You're writing a report in Microsoft Office while listening to online music from
    iTunes.

    This is an example of multiple processes. Office and iTunes don't need to share
    memory and are different applications running different code, so they are run as
    separate processes.


\subsection{What is a race condition?}
A race condition occurs when two or more threads attempt to access the same resource,
such as a variable, at the same time. This can result in unexpected behaviour such as
one thread overwriting the value written by another thread while the other thread is
still using the value.

    \subsubsection{What is a mutex?}
    A mutex ensures that only one thread can access a resource at a time.
    Before a thread accesses a shared resource, it must first lock the mutex associated with it.
    If the mutex is already locked, the thread will wait until it is unlocked.
    Once the thread is finished with the resource, it unlocks the mutex so that other
    threads can access it.

    \subsubsection{What is a semaphore and how does it differ?}
    A semaphore is similar to a mutex, except that it allows one or more threads to hold
    a lock at a time instead of just one. A semaphore is initialised with a value
    that indicates the number of locks that can be held at a time.

    Similar to a mutex, a thread must first lock the semaphore before accessing the shared resource.
    If there are no locks available, the thread will wait until one becomes available.
    The thread then unlocks the semaphore when it is finished with the resource.
    The lock count is decremented by one when the semaphore is locked and incremented
    when it is unlocked.

    Semaphores can be used instead of mutexes when a resource has limited capacity,
    such as with I/O devices, but can still be safely accessed by multiple threads at the same time.

\subsection{What happens during a context switch?}
// TODO: This

\subsection{What is TLS}
// TODO: This
    \subsubsection{How does TLS protect private information?}
    // TODO: This

\section{The Lunar Lander Controller}{
    % https://tex.stackexchange.com/questions/74529/sections-indexed-with-numbers-subsections-with-letters
    \renewcommand{\thesubsection}{\thesection.\alph{subsection}}

    \subsection{Threads and Semaphores in Code}
    // TODO: This

    \subsection{Advantages and disadvantages of UDP nad TCP}
    // TODO: This

    \subsection{Data Logging}
        // TODO: This

        \subsubsection{What data is logged and why?}
        // TODO: This

        \subsubsection{What are the advantages and disadvantages of logging five times per second?}
        // TODO: This

        \subsubsection{How often would data be logged if not a fixed interval?}
        // TODO: This

        \subsubsection{{What rate does the size of the file grow and why?}}
        // TODO: This
}

\section*{References}
List of references.

\section*{Appendix}

\subsection*{controller.c}
\lstinputlisting{src/controller.c}

\subsection*{libnet.c}
\lstinputlisting{src/libnet.c}

\end{document}
